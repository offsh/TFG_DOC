\newglossaryentry{Malware}
{
    name=Malware,
    description={O software  `malicioso' es todo aquél programa o código que pretende (de forma intencionada) causar daños y/o sacar beneficios de un sistema}
}

\newglossaryentry{cloud}
{
    name=Cloud,
    description={La computación en la nube o \textit{cloud} (del inglés cloud computing), conocida también como servicios en la nube, informática en la nube, nube de cómputo o simplemente «la nube», es un paradigma que permite ofrecer servicios de computación a través de una red, que usualmente es internet}
}

\newglossaryentry{Ransomware}
{
    name=Ransomware,
    description={Es un tipo de \gls{Malware} que hace públicos o inaccesibles (por medio de encriptación, por ejemplo) los datos de la víctima con el objetivo de chantajearla para que pague un `rescate'}
}


\newglossaryentry{Reverse engineering}
{
    name={ingeniería inversa},
    description={Es una técnica que consiste en tratar de obtener por medios deductivos información sobre un producto o sistema haciendo uso del mismo y tratando de figurar como está diseñado}
}

\newglossaryentry{obfuscation}{
    name={obfuscation},
    description={Ofuscación, ocultación, anonimato. Es el acto de evitar ser descubierto mientras se realiza un ataque o una auditoría. Bien eliminando los rastros que se puedan dejar u ocultándolos por medio de falsos rastros que los escodan}
}

\newglossaryentry{network sniffing}
{
    name=network sniffing,
    description={Es la acción de atrapar y atender a todo el tráfico indiscriminado que circula por una red, sea cual sea su destinatario, con el objetivo de obtener algún tipo de información de interés}
}

\newglossaryentry{Rolling Release}
{
    name=Rolling Release,
    description={Es un tipo de distribución de Software en el que las actualizaciones son continuas en lugar de depender de un versionado discreto. Los cambios se van añadiendo de forma incremental conforme van siendo disponibles en lugar de ir emitiendo nuevas versiones con todos los cambios desde la anterior}
}

\newglossaryentry{compliance} 
{
    name=compliance,
    description={'el cumplimiento' de las normativas o leyes referentes a la seguridad de los datos que una empresa pueda almacenar o gestionar}
}

\newglossaryentry{vagrant}
{
    name=Vagrant,
    description={'una herramienta diseñada para el despliegue y configuración de entornos de máquinas virtuales (utilizando diversos proveedores como virtualbox, qemu, aws, etc}
}

\newacronym{ddos}{DDOS}{Distributed Denial Of Service}

\newacronym{aws}{AWS}{Amazon Web Service}


\newacronym{iot}{IOT}{The Internet of Things}

\newglossaryentry{OpenSource}
{
    name=OpenSource,
    description={OpenSource o código abierto es un tipo de software liberado con una licencia que asegura el derecho de los usuarios a usar, estudiar, cambiar y distribuir el mismo con cualquier propósito}
}

\newglossaryentry{Pull Request}
{
    name=Pull Request,
    description={Una Pull Request es la acción de validar un código que se va a mergear de una rama a otra. Por ejemplo, de una rama de desarrollo en un Fork de un proyecto a una rama oficial}
}

\newglossaryentry{Fuzzing}
{
    name=Fuzzing,
    description={Según la OWASP, fuzzing es el acto de introducir datos mal formados en un programa con el objetivo de conseguir un comportamiento en este inesperado. Aplicado, por ejemplo, al ámbito de web, podríamos considerar fuzzing las técnicas de SQL injection y similares, dónde se introduce código SQL en lugares como los credenciales de acceso para conseguir logearse con un usuario distinto al que poseemos}
}

\newglossaryentry{syslog}
{
    name=Syslog,
    description={syslog es un estándar para el envío de mensajes de registro (logs) en una red informática IP. Por syslog se conoce tanto al protocolo de red como a la aplicación o biblioteca que envía los mensajes de registro. Un mensaje de registro suele tener información sobre la seguridad del sistema, aunque puede contener cualquier información. Junto con cada mensaje se incluye la fecha y hora del envío}
}

\newglossaryentry{phishing}
{
    name=Phishing,
    description={Técnica empleada por delincuente cibernéticos para estafar y obtener información de sus víctimas haciéndose pasar por otra persona o entidad (a través de correo electrónico, redes sociales, etc}
}


\newacronym{IaC}{IAC}{Infrastructure as Code}
\newacronym{CPA}{CPA}{Certified Public Accountant}
\newacronym{FOSS}{FOSS}{Free & Open-Source Software}

\newglossaryentry{forensics}
{
    name=forensics,
    description={Es el término que describe a las acciones llevadas a cabo para recopilar información de los sistemas informáticos que pueda ser utilizada para demostrar hechos por ejemplo durante una investigación relacionada con un crimen cibernético}
}

\newglossaryentry{antiforensics}
{
    name=anti-forensics,
    description={Técnicismo que se emplea para referise a aquellas acciones llevadas a cabo para dificultar las labores de investigación forense (forensics \gls{forensics}) de los equipos de seguridad de una empresa. Borrar u ocultar los rastros que se pueda haber dejado durante la explotación de vulnerabilidades de un sistema con el objetivo de imposibilitar el descubrimiento del mismo por parte de los administradores del sistema}
}

\newglossaryentry{siem}
{
    name=SIEM,
    description={`Security Information and Event Management', es un tipo de software que permite recopilar y analizar información de seguridad de distintos dispositivos, alertando al usuario de aquellos eventos importantes que tengan lugar}
}


\newglossaryentry{HIDS}
{
    name=HIDS,
    description={`Host Based Intrussion Detection System', es un tipo de software que permite detertar intentos de intrusiones en un sistema y reportarlas}
}

\newglossaryentry{mitre}
{
    name=MITRE ATT\&ACK,
    description={Una base de datos pública de conocimiento de tácticas de `ataques adversarios' que trata de modelar y agrupar las distintas acciones que puede llevar a cabo un criminar para dañar a una entidad}
}

\newglossaryentry{fork}
{
    name=fork,
    description={También llamado `bifurcación' en español. Es el desarrollo de un proyecto informático tomando como base el código fuente de uno ya existente o de alguna ramificación de este. Un ejemplo claro de esto son las distintas diburcaciones de desarrollo de las distribuciones de Linux, dónde Ubuntu, por ejemplo, es una bifurcación o un fork de Debian}
}

\newglossaryentry{reverse_shell}
{
    name=Consola inversa,
    description={O en inglés `reverse shell'. Es un tipo de conexión entre dos hosts que ocurre de forma opuesta a la habitual: desde el dispositivo que se va a conectar se abre una aplicación que `espera' una conexión y desde el dispositivo que va a ser accedido se abre una consola de comandos que se conecta a dicha aplicación. Es una técnica que se utiliza en el ámbito de la ciberseguridad para conseguir acceso remoto a dispositivos dónde podemos ejecutar código arbitrario de algún modo}
}

\newglossaryentry{bugb}
{
    name=Bug Bounty,
    description={Son programas en los que empresas u organizaciones ofrecen importantes recompensas económicas a aquellas personas (ajenas a la organización) capaces de encontrar vulnerabilidades o errores de seguridad en alguno de sus sistemas.}
}